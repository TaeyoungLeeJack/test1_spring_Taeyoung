\documentclass{article}

\usepackage{pgfplots}
\usepackage[margin=0.75in, paperwidth=8.5in, paperheight=11in]{geometry}
\usepackage{setspace}
\usepackage{fancyvrb} % extended verbatim environments
\usepackage{framed}%To get shade behind text
\usepackage{marginnote}
\definecolor{shadecolor}{rgb}{0.9,0.9,0.9}%setting shade color


\begin{document}
\pagenumbering{gobble}

\doublespacing
\textbf{IB Computer Science }                        %%%(class number and section) 
 \hfill                             %%%(date of test)
$ {\bf Name:Taeyoung Lee } \underline{\hspace{2.5in}}$(3 points)

\begin{centering}
\vspace{1cm}
\textbf{Spring: Exam 1}\\
\end{centering}
\vspace{1cm}
 

  
 
 $\bf{1)}$ Write a Java class that prints the following. (10 points)
  
   \begin{verbatim}
 Hola mundo!
 \end{verbatim}

 \marginnote{Write the class Hola and Write the main method that prints "Hola mundo!".}

   \begin{verbatim}
   public class Hola 
   public static avoid main(Stirng[]args)
   System.out.println("Hola mundo")
   \end{verbatim}
 
 $\bf{2)}$ Write a method named "counter" that takes two integers, A and B, and prints the numbers from A to B. (20 points)
  \vspace{0.5cm}

\marginnote{Make a method that contains a and b integers, and make a for loop that starts to count one up from a by setting i=a until exact b and if it is, print all the i. Else if a is larger than b, count one down until a is as same as b. Then if it's done, print all the i.}

 \begin{verbatim}
 public static avoid counter(int A,int B)
 if (a<b) 
 for (int 0 i=a;i<=b;i++){
 System.out.println(i);}
 
 \end{verbatim}
 
  $\bf{3)}$ Write some Java code that will fill an array with the numbers from 10 to 100.  (20 points)
   \vspace{0.5cm}

\marginnote{The array from 10 to 100 has a length of 91. Therefore define the new integer array that has length of 91, and then make a for loop counting one up from 0 to less than 91. Then add 10 to every elements of the array, so can prints from 10.}

   \begin{verbatim}
   int[] non=new int[91];
   for(int i=0;i<91;i++){
   non[i]=i+10; }
   \end{verbatim}
   
  $\bf{4)}$ Write a method named "average" that will return the average value of an integer array. \\
   It should return a double.  (20 points)
   \vspace{0.5cm}

\marginnote{make a method that is static and able to return. Under that, define an integer sum=0 and add every element of the array to the sum until its length of the array. And then the new sum is divided by its length of the array and by making defining as a float make the number float.}

   \begin{verbatim}
   public static double average(int[] ints)
   int sum=0
   for(int i=0,;< ints.length; i++){
   interger sum=sum+name[i];}
   return(float)sum)name.length;
   \end{verbatim}
   
  $\bf{5)}$ Draw the truth table for OR and XOR.  (10 points)
   \vspace{0.5cm}
\begin{verbatim}
Or  T   F      Xor  T   F
T    T   T      T     F   T
F     T   F      F     T    F
\end{verbatim}
   
  $\bf{6)}$ Write a method for the XOR operator named "xor".  It should take two booleans as arguments and return a boolean.  (20 points)
   \vspace{0.5cm}
  \marginnote{According to the table Xor is true when only inputs are different. Otherwise it results false. Threrefore write a method that is able to return and define boolean A and B. Then if A and B are same, return false else return true. }
   \begin{verbatim}
   public static boolean,xor(boolean a,boolean b){
   if (a==b) {return false}
   else {return true}
   return (a&&b)) &&(a,b)
   \end{verbatim}
   $\bf{7)}$ Explain how to compile and run a program "hello.java" from the command line.  (10 points)
  \begin{verbatim}
  java hello. java > compile that generates hello.class file
  java hello > run the code
  \end{verbatim}


  
  
  
    

 
\end{document}